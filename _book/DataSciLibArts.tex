\documentclass[]{book}
\usepackage{lmodern}
\usepackage{amssymb,amsmath}
\usepackage{ifxetex,ifluatex}
\usepackage{fixltx2e} % provides \textsubscript
\ifnum 0\ifxetex 1\fi\ifluatex 1\fi=0 % if pdftex
  \usepackage[T1]{fontenc}
  \usepackage[utf8]{inputenc}
\else % if luatex or xelatex
  \ifxetex
    \usepackage{mathspec}
  \else
    \usepackage{fontspec}
  \fi
  \defaultfontfeatures{Ligatures=TeX,Scale=MatchLowercase}
\fi
% use upquote if available, for straight quotes in verbatim environments
\IfFileExists{upquote.sty}{\usepackage{upquote}}{}
% use microtype if available
\IfFileExists{microtype.sty}{%
\usepackage{microtype}
\UseMicrotypeSet[protrusion]{basicmath} % disable protrusion for tt fonts
}{}
\usepackage[margin=1in]{geometry}
\usepackage{hyperref}
\hypersetup{unicode=true,
            pdftitle={Data Science for the Liberal Arts},
            pdfauthor={Kevin Lanning},
            pdfborder={0 0 0},
            breaklinks=true}
\urlstyle{same}  % don't use monospace font for urls
\usepackage{natbib}
\bibliographystyle{apalike}
\usepackage{longtable,booktabs}
\usepackage{graphicx,grffile}
\makeatletter
\def\maxwidth{\ifdim\Gin@nat@width>\linewidth\linewidth\else\Gin@nat@width\fi}
\def\maxheight{\ifdim\Gin@nat@height>\textheight\textheight\else\Gin@nat@height\fi}
\makeatother
% Scale images if necessary, so that they will not overflow the page
% margins by default, and it is still possible to overwrite the defaults
% using explicit options in \includegraphics[width, height, ...]{}
\setkeys{Gin}{width=\maxwidth,height=\maxheight,keepaspectratio}
\usepackage[normalem]{ulem}
% avoid problems with \sout in headers with hyperref:
\pdfstringdefDisableCommands{\renewcommand{\sout}{}}
\IfFileExists{parskip.sty}{%
\usepackage{parskip}
}{% else
\setlength{\parindent}{0pt}
\setlength{\parskip}{6pt plus 2pt minus 1pt}
}
\setlength{\emergencystretch}{3em}  % prevent overfull lines
\providecommand{\tightlist}{%
  \setlength{\itemsep}{0pt}\setlength{\parskip}{0pt}}
\setcounter{secnumdepth}{5}
% Redefines (sub)paragraphs to behave more like sections
\ifx\paragraph\undefined\else
\let\oldparagraph\paragraph
\renewcommand{\paragraph}[1]{\oldparagraph{#1}\mbox{}}
\fi
\ifx\subparagraph\undefined\else
\let\oldsubparagraph\subparagraph
\renewcommand{\subparagraph}[1]{\oldsubparagraph{#1}\mbox{}}
\fi

%%% Use protect on footnotes to avoid problems with footnotes in titles
\let\rmarkdownfootnote\footnote%
\def\footnote{\protect\rmarkdownfootnote}

%%% Change title format to be more compact
\usepackage{titling}

% Create subtitle command for use in maketitle
\newcommand{\subtitle}[1]{
  \posttitle{
    \begin{center}\large#1\end{center}
    }
}

\setlength{\droptitle}{-2em}
  \title{Data Science for the Liberal Arts}
  \pretitle{\vspace{\droptitle}\centering\huge}
  \posttitle{\par}
  \author{Kevin Lanning}
  \preauthor{\centering\large\emph}
  \postauthor{\par}
  \predate{\centering\large\emph}
  \postdate{\par}
  \date{2018-01-06}

\usepackage{booktabs}

\usepackage{amsthm}
\newtheorem{theorem}{Theorem}[chapter]
\newtheorem{lemma}{Lemma}[chapter]
\theoremstyle{definition}
\newtheorem{definition}{Definition}[chapter]
\newtheorem{corollary}{Corollary}[chapter]
\newtheorem{proposition}{Proposition}[chapter]
\theoremstyle{definition}
\newtheorem{example}{Example}[chapter]
\theoremstyle{definition}
\newtheorem{exercise}{Exercise}[chapter]
\theoremstyle{remark}
\newtheorem*{remark}{Remark}
\newtheorem*{solution}{Solution}
\begin{document}
\maketitle

{
\setcounter{tocdepth}{1}
\tableofcontents
}
\hypertarget{part-part-i-introduction}{%
\part{Part I Introduction}\label{part-part-i-introduction}}

\hypertarget{preface}{%
\chapter{Preface}\label{preface}}

This work-in-progress includes the notes for \emph{\emph{Introduction to
Data Science}} at the Wilkes Honors College of Florida Atlantic
University. It's written using the R package Bookdown
\citep{R-bookdown}. The source code for the book may be found at, and
the class syllabus and schedule may be found at.

\hypertarget{introduction}{%
\chapter{Introduction}\label{introduction}}

Hochster (in
\href{https://arxiv.org/ftp/arxiv/papers/1612/1612.07140.pdf}{Hicks \&
Irizarry, 2017}) describes two broad types of data scientists: Type A
(Analysis) data scientists, whose skills are like those of an applied
\textbf{statistician}, and Type B (Building) data scientists, whose
skills lie in problem solving or coding, using the skills of the
\textbf{computer scientist}. Our course is like those at the
universities of \href{https://idc9.github.io/stor390/}{North Carolina},
\href{https://github.com/STAT545-UBC/STAT545-UBC.github.io}{British
Columbia},
\href{https://www2.stat.duke.edu/courses/Fall15/sta112.01/}{Duke},
\href{http://www.hcbravo.org/IntroDataSci/calendar/}{Maryland},
\href{http://pages.stat.wisc.edu/~yandell/R_for_data_sciences/syllabus.html}{Wisconsin},
\href{https://github.com/dcl-2017-04/curriculum}{Stanford},
\href{https://byuistats.github.io/M335/syllabus.html}{BYU},
\href{http://datasciencelabs.github.io/}{Harvard},
\href{https://github.com/MUSA-620-Spring-2017/Course-Materials}{Pennsylvania},
and \href{https://github.com/FAUDataScience/stat259}{UC Berkeley} (and
will likely draw from all of these) in that it is closer to a Type A
than a Type B treatment, one which is closer to Statistics than to
Computer Science. But there's more.

\hypertarget{type-c-data-science}{%
\section{Type C data science}\label{type-c-data-science}}

Hochster's view of data science arguably omits a critical component of
the field. Data science is driven not just by statistics and computer
science, but also by ``domain expertise:''

The iconic
\href{https://www.google.com/search?q=venn+diagram+model+of+data+science\&newwindow=1\&safe=active\&rlz=1C1CHBF_enUS762US763\&tbm=isch\&tbo=u\&source=univ\&sa=X\&ved=0ahUKEwiM_abBtY7XAhXDQCYKHdgyB58QsAQIOg\&biw=1378}{Venn
diagram model of data science} suggests what we can call a ``Type C data
science.'' It begins with ``domain expertise'' in your
\textbf{concentration} in the arts, humanities, social and/or natural
sciences, it both informs and can be informed by new methods and tools
of data analysis, and it includes such things as \textbf{communication}
(including writing and the design and display of quantitative data),
\textbf{collaboration} (making use of the tools of team science), and
\textbf{citizenship} (serving the public good, overcoming the digital
divide, furthering social justice, increasing public health, diminishing
human suffering, and making the world a more beautiful place). It's
shaped, too, by an awareness of the \textbf{creepiness} of living
increasingly in a measured, observed world.

\textbf{The WHC Intro to Data Science course will be a Type C course, or
more accurately, a CAB course - equal parts statistics and domain
knowledge, with just enough computing to be proficient to use (but not
yet build) the software tools at our disposal.} We aren't unique here -
there are courses (which we may again draw from) with similar goals at
schools including
\href{https://github.com/UC-MACSS/persp-analysis}{Chicago},
\href{https://github.com/jacobeisenstein/gt-css-class}{Georgia Tech},
\href{https://github.com/raviolli77/dataScience-UCSBProjectGroup-Syllabus}{UC
Santa Barbara},
\href{http://www.princeton.edu/~mjs3/soc596_f2016/}{Princeton},
\href{https://github.com/rochelleterman/PS239T}{UC Berkeley}, at
\href{https://github.com/HertieDataScience/SyllabusAndLectures}{Berlin's
Hertie School of Governance}, and in
\href{https://github.com/tommeagher/data1-fall2015}{Columbia's School of
Journalism}.

\hypertarget{what-will-be-in-the-class}{%
\section{What will be in the class?}\label{what-will-be-in-the-class}}

\textbf{R}

In my rough survey of introductory data science courses, I saw a pretty
even split between those which begin with Python and those which begin
with the statistical programming language R. This difference
corresponds, loosely, to the split noted above: Computer science based
approaches to data science are frequently grounded in Python, while
stats based approaches are generally grounded in R. Our course, like
those for most of the syllabi and courses linked above, will be based in
R.

\textbf{Reproducible science}

The course will provide an introduction to some of the methods and tools
of reproducible science. We will consider the replication crisis in the
natural and social sciences, and then consider three distinct approaches
which serve as partial solutions to the crisis. The first of these is
training in a notebook-based approach to writing analyses, reports and
projects (using R markdown). The second is using public repositories
(such as the \href{https://osf.io/}{Open Science Framework} and
\href{https://github.com/}{GitHub}) to provide snapshots of projects
over time. Finally, the third is to consider the place of significance
testing in the age of Big Data, and to provide training in the use of
descriptive, exploratory techniques of data analysis.

\textbf{Good visualizations}

Part of Type C data science is communication, and this includes not just
writing up results, but also designing data displays that incisively
convey the key ideas or features in a flood of data. We'll examine and
develop data visualizations such as plots, networks and text clouds.
More advanced topics may include maps, interactive displays, and
animations.

\textbf{\sout{All} \emph{Some of} the data}

It's been
\href{https://www.udemy.com/datascience/learn/v4/t/lecture/3473822?start=379}{argued}
that in the last dozen years, humans have produced more than 60 times as
much information as existed in the entire previous history of humankind.
(It sounds like hyperbole, but even if it's off by an order of magnitude
it's still amazing). There are plenty of data sources for us to examine,
and we'll consider existing datasets from disciplines ranging from
literature to economics to public health, with sizes ranging from a few
dozen to millions of data points. We will also clean and create new
datasets.

\textbf{\sout{All} \emph{Some of} the meaning}

Data matter, can save, enhance, or destroy human lives. (This is a
crummy sentence: My pedantic insistence on treating the term ``data'' as
plural rather than singular likely distracted you from the substance of
my message. I'll leave it here as a reminder that we can all become
better writers). Back to my point: In this class, we'll explore
approaches to analyzing the meaning of data in areas including the
analyses of simple texts and data journalism.

\textbf{\sout{All} \emph{Some of} the skills}

The skills required to extract meaning from data include an
understanding of classical statistical principles (e.g., probability
theory and sampling theory), core statistical techniques (regression),
and the extension of these core principles and basic techniques to
problems in natural language processing, the analysis of social
networks, and machine learning and classification.

\textbf{\sout{All} \emph{Some of} the tools}

In addition to R, we'll use a range of other tools: We'll communicate on
the Slack platform. We'll write using markdown editors such as
\href{https://typora.io/}{Typora}. We'll certainly use spreadsheets such
as Excel or Google Sheets. We \emph{may} use additional tools for
visualizing data such as Gephi and Tableau.

\textbf{Hands-on computing}

We had initially anticipated that the lectures would include discussion,
and that the computing part of the class would occur just in the lab.
But, in the course of examining syllabi at other schools, it became
apparent to me that \textbf{there will be computing throughout the
course}, not just in the lab.

\textbf{What will be in the lab?}

The labs will have two features. First, they will allow for a
project-based approach, focused on the collaborative analysis of
problems of your own choosing. Second, these projects will include a
deeper dive into some of the topics and problems above. Here are a few
examples of how the treatment in the lecture and the lab are likely to
differ:

\begin{longtable}[]{@{}ccc@{}}
\toprule
\begin{minipage}[b]{0.30\columnwidth}\centering
Topic\strut
\end{minipage} & \begin{minipage}[b]{0.30\columnwidth}\centering
Lecture\strut
\end{minipage} & \begin{minipage}[b]{0.30\columnwidth}\centering
Lab: Deeper dive\strut
\end{minipage}\tabularnewline
\midrule
\endhead
\begin{minipage}[t]{0.30\columnwidth}\centering
Introduction\strut
\end{minipage} & \begin{minipage}[t]{0.30\columnwidth}\centering
Stephens-Davidowitz book; Google trends\strut
\end{minipage} & \begin{minipage}[t]{0.30\columnwidth}\centering
Examining one or more scholarly papers in data journalism, computational
science, or computational social science\strut
\end{minipage}\tabularnewline
\begin{minipage}[t]{0.30\columnwidth}\centering
Getting data\strut
\end{minipage} & \begin{minipage}[t]{0.30\columnwidth}\centering
Extant data\strut
\end{minipage} & \begin{minipage}[t]{0.30\columnwidth}\centering
Using APIs to scrape social media\strut
\end{minipage}\tabularnewline
\begin{minipage}[t]{0.30\columnwidth}\centering
Sharing science\strut
\end{minipage} & \begin{minipage}[t]{0.30\columnwidth}\centering
Setting up account on OSF\strut
\end{minipage} & \begin{minipage}[t]{0.30\columnwidth}\centering
Using GitHub, Becoming a Repo \sout{Man} \emph{person}\strut
\end{minipage}\tabularnewline
\begin{minipage}[t]{0.30\columnwidth}\centering
Exploratory data analysis / Data visualization\strut
\end{minipage} & \begin{minipage}[t]{0.30\columnwidth}\centering
Static data displays\strut
\end{minipage} & \begin{minipage}[t]{0.30\columnwidth}\centering
Interactive plots (r Shiny), animated displays?, Tableau?\strut
\end{minipage}\tabularnewline
\begin{minipage}[t]{0.30\columnwidth}\centering
Sampling theory\strut
\end{minipage} & \begin{minipage}[t]{0.30\columnwidth}\centering
Test vs training datasets\strut
\end{minipage} & \begin{minipage}[t]{0.30\columnwidth}\centering
k-fold cross-validation\strut
\end{minipage}\tabularnewline
\begin{minipage}[t]{0.30\columnwidth}\centering
Regression and classification\strut
\end{minipage} & \begin{minipage}[t]{0.30\columnwidth}\centering
Linear and logistic regression\strut
\end{minipage} & \begin{minipage}[t]{0.30\columnwidth}\centering
Machine learning: Robust techniques / regularized regression Supervised
prediction, possibly semi-supervised and unsupervised regression\strut
\end{minipage}\tabularnewline
\begin{minipage}[t]{0.30\columnwidth}\centering
Graph theory and network analysis\strut
\end{minipage} & \begin{minipage}[t]{0.30\columnwidth}\centering
Introduction to centrality, community structure, contagion\strut
\end{minipage} & \begin{minipage}[t]{0.30\columnwidth}\centering
Network robustness, different approaches to centrality \& community
structure\strut
\end{minipage}\tabularnewline
\begin{minipage}[t]{0.30\columnwidth}\centering
Analyzing texts\strut
\end{minipage} & \begin{minipage}[t]{0.30\columnwidth}\centering
Word clouds\strut
\end{minipage} & \begin{minipage}[t]{0.30\columnwidth}\centering
Text mining, natural language analysis\strut
\end{minipage}\tabularnewline
\begin{minipage}[t]{0.30\columnwidth}\centering
Generating products\strut
\end{minipage} & \begin{minipage}[t]{0.30\columnwidth}\centering
Team project in class\strut
\end{minipage} & \begin{minipage}[t]{0.30\columnwidth}\centering
Project in class and poster\strut
\end{minipage}\tabularnewline
\bottomrule
\end{longtable}

\bibliography{dataSciRefs.bib}


\end{document}
